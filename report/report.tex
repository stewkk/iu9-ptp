% !TeX TXS-program:bibliography = txs:///biber
\documentclass[14pt, russian]{scrartcl}
\let\counterwithout\relax
\let\counterwithin\relax
\usepackage{float}
\usepackage{xcolor}
\usepackage{extsizes}
\usepackage{subfig}
\usepackage[export]{adjustbox}
\usepackage{tocvsec2} % возможность менять учитываемую глубину разделов в оглавлении
\usepackage[subfigure]{tocloft}
\usepackage[newfloat]{minted}
\captionsetup[listing]{position=top}

\makeatletter
\AddToHook{begindocument/before}{\@ifpackageloaded{minted}{\removefromtoclist[float]{lol}}{}}
\makeatother

\AtBeginEnvironment{figure}{\vspace{0.5cm}}
\AtBeginEnvironment{table}{\vspace{0.5cm}}
\AtBeginEnvironment{listing}{\vspace{0.5cm}}
\AtBeginEnvironment{algorithm}{\vspace{0.5cm}}
\AtBeginEnvironment{minted}{\vspace{-0.5cm}}

\usepackage{fancyvrb}
\usepackage{ulem,bm,mathrsfs,ifsym} %зачеркивания, особо жирный стиль и RSFS начертание
\usepackage{sectsty} % переопределение стилей подразделов
%%%%%%%%%%%%%%%%%%%%%%%

%%% Поля и разметка страницы %%%
\usepackage{pdflscape}                              % Для включения альбомных страниц
\usepackage{geometry}                               % Для последующего задания полей
\geometry{a4paper,tmargin=2cm,bmargin=2cm,lmargin=3cm,rmargin=1cm} % тоже самое, но лучше

%%% Математические пакеты %%%
\usepackage{amsthm,amsfonts,amsmath,amssymb,amscd}  % Математические дополнения от AMS
\usepackage{mathtools}                              % Добавляет окружение multlined
\usepackage[perpage]{footmisc}
%\usepackage{times}

%%%% Установки для размера шрифта 14 pt %%%%
%% Формирование переменных и констант для сравнения (один раз для всех подключаемых файлов)%%
%% должно располагаться до вызова пакета fontspec или polyglossia, потому что они сбивают его работу
%\newlength{\curtextsize}
%\newlength{\bigtextsize}
%\setlength{\bigtextsize}{13pt}
\KOMAoptions{fontsize=14pt}

\makeatletter
\def\showfontsize{\f@size{} point}
\makeatother

\makeatletter
\setlength{\@fptop}{0pt}
\makeatother

%\makeatletter
%\show\f@size                                       % неплохо для отслеживания, но вызывает стопорение процесса, если документ компилируется без команды  -interaction=nonstopmode
%\setlength{\curtextsize}{\f@size pt}
%\makeatother

%шрифт times
\usepackage{tempora}
%\usepackage{pscyr}
%\setmainfont[Ligatures={TeX,Historic}]{Times New Roman}

   %%% Решение проблемы копирования текста в буфер кракозябрами
%    \input glyphtounicode.tex
%    \input glyphtounicode-cmr.tex %from pdfx package
%    \pdfgentounicode=1
    \usepackage{cmap}                               % Улучшенный поиск русских слов в полученном pdf-файле
    \usepackage[T1]{fontenc}                       % Поддержка русских букв
    \usepackage[utf8]{inputenc}                     % Кодировка utf8
    \usepackage[english, main=russian]{babel}            % Языки: русский, английский
%   \IfFileExists{pscyr.sty}{\usepackage{pscyr}}{}  % Красивые русские шрифты
%\renewcommand{\rmdefault}{ftm}
%%% Оформление абзацев %%%
\usepackage{indentfirst}                            % Красная строка
%\usepackage{eskdpz}

%%% Таблицы %%%
\usepackage{longtable}                              % Длинные таблицы
\usepackage{multirow,makecell,array}                % Улучшенное форматирование таблиц
\usepackage{booktabs}                               % Возможность оформления таблиц в классическом книжном стиле (при правильном использовании не противоречит ГОСТ)

%%% Общее форматирование
\usepackage{soulutf8}                               % Поддержка переносоустойчивых подчёркиваний и зачёркиваний
\usepackage{icomma}                                 % Запятая в десятичных дробях



%%% Изображения %%%
\usepackage{graphicx}                               % Подключаем пакет работы с графикой
\usepackage{wrapfig}

%%% Списки %%%
\usepackage{enumitem}

%%% Подписи %%%
\usepackage{caption}                                % Для управления подписями (рисунков и таблиц) % Может управлять номерами рисунков и таблиц с caption %Иногда может управлять заголовками в списках рисунков и таблиц
%% Использование:
%\begin{table}[h!]\ContinuedFloat - чтобы не переключать счетчик
%\captionsetup{labelformat=continued}% должен стоять до самого caption
%\caption{}
% либо ручками \caption*{Продолжение таблицы~\ref{...}.} :)

%%% Интервалы %%%
\addto\captionsrussian{%
  \renewcommand{\listingname}{Листинг}%
}
%%% Счётчики %%%
\usepackage[figure,table,section]{totalcount}               % Счётчик рисунков и таблиц
\DeclareTotalCounter{lstlisting}
\usepackage{totcount}                               % Пакет создания счётчиков на основе последнего номера подсчитываемого элемента (может требовать дважды компилировать документ)
\usepackage{totpages}                               % Счётчик страниц, совместимый с hyperref (ссылается на номер последней страницы). Желательно ставить последним пакетом в преамбуле

%%% Продвинутое управление групповыми ссылками (пока только формулами) %%%
%% Кодировки и шрифты %%%

%   \newfontfamily{\cyrillicfont}{Times New Roman}
%   \newfontfamily{\cyrillicfonttt}{CMU Typewriter Text}
	%\setmainfont{Times New Roman}
	%\newfontfamily\cyrillicfont{Times New Roman}
	%\setsansfont{Times New Roman}                    %% задаёт шрифт без засечек
%	\setmonofont{Liberation Mono}               %% задаёт моноширинный шрифт
%    \IfFileExists{pscyr.sty}{\renewcommand{\rmdefault}{ftm}}{}
%%% Интервалы %%%
%linespread-реализация ближе к реализации полуторного интервала в ворде.
%setspace реализация заточена под шрифты 10, 11, 12pt, под остальные кегли хуже, но всё же ближе к типографской классике.
\linespread{1.3}                    % Полуторный интервал (ГОСТ Р 7.0.11-2011, 5.3.6)
%\renewcommand{\@biblabel}[1]{#1}

%%% Гиперссылки %%%
\usepackage{hyperref}

%%% Выравнивание и переносы %%%
\sloppy                             % Избавляемся от переполнений
\clubpenalty=10000                  % Запрещаем разрыв страницы после первой строки абзаца
\widowpenalty=10000                 % Запрещаем разрыв страницы после последней строки абзаца

\makeatletter % малые заглавные, small caps shape
\let\@@scshape=\scshape
\renewcommand{\scshape}{%
  \ifnum\strcmp{\f@series}{bx}=\z@
    \usefont{T1}{cmr}{bx}{sc}%
  \else
    \ifnum\strcmp{\f@shape}{it}=\z@
      \fontshape{scsl}\selectfont
    \else
      \@@scshape
    \fi
  \fi}
\makeatother

%%% Подписи %%%
%\captionsetup{%
%singlelinecheck=off,                % Многострочные подписи, например у таблиц
%skip=2pt,                           % Вертикальная отбивка между подписью и содержимым рисунка или таблицы определяется ключом
%justification=centering,            % Центрирование подписей, заданных командой \caption
%}
%%%        Подключение пакетов                 %%%
\usepackage{ifthen}                 % добавляет ifthenelse
%%% Инициализирование переменных, не трогать!  %%%
\newcounter{intvl}
\newcounter{otstup}
\newcounter{contnumeq}
\newcounter{contnumfig}
\newcounter{contnumtab}
\newcounter{pgnum}
\newcounter{bibliosel}
\newcounter{chapstyle}
\newcounter{headingdelim}
\newcounter{headingalign}
\newcounter{headingsize}
\newcounter{tabcap}
\newcounter{tablaba}
\newcounter{tabtita}
%%%%%%%%%%%%%%%%%%%%%%%%%%%%%%%%%%%%%%%%%%%%%%%%%%

%%% Область упрощённого управления оформлением %%%

%% Интервал между заголовками и между заголовком и текстом
% Заголовки отделяют от текста сверху и снизу тремя интервалами (ГОСТ Р 7.0.11-2011, 5.3.5)
\setcounter{intvl}{3}               % Коэффициент кратности к размеру шрифта

%% Отступы у заголовков в тексте
\setcounter{otstup}{0}              % 0 --- без отступа; 1 --- абзацный отступ

%% Нумерация формул, таблиц и рисунков
\setcounter{contnumeq}{1}           % Нумерация формул: 0 --- пораздельно (во введении подряд, без номера раздела); 1 --- сквозная нумерация по всей диссертации
\setcounter{contnumfig}{1}          % Нумерация рисунков: 0 --- пораздельно (во введении подряд, без номера раздела); 1 --- сквозная нумерация по всей диссертации
\setcounter{contnumtab}{1}          % Нумерация таблиц: 0 --- пораздельно (во введении подряд, без номера раздела); 1 --- сквозная нумерация по всей диссертации

%% Оглавление
\setcounter{pgnum}{0}               % 0 --- номера страниц никак не обозначены; 1 --- Стр. над номерами страниц (дважды компилировать после изменения)

%% Библиография
\setcounter{bibliosel}{1}           % 0 --- встроенная реализация с загрузкой файла через движок bibtex8; 1 --- реализация пакетом biblatex через движок biber

%% Текст и форматирование заголовков
\setcounter{chapstyle}{1}           % 0 --- разделы только под номером; 1 --- разделы с названием "Глава" перед номером
\setcounter{headingdelim}{1}        % 0 --- номер отделен пропуском в 1em или \quad; 1 --- номера разделов и приложений отделены точкой с пробелом, подразделы пропуском без точки; 2 --- номера разделов, подразделов и приложений отделены точкой с пробелом.

%% Выравнивание заголовков в тексте
\setcounter{headingalign}{0}        % 0 --- по центру; 1 --- по левому краю

%% Размеры заголовков в тексте
\setcounter{headingsize}{0}         % 0 --- по ГОСТ, все всегда 14 пт; 1 --- пропорционально изменяющийся размер в зависимости от базового шрифта

%% Подпись таблиц
\setcounter{tabcap}{0}              % 0 --- по ГОСТ, номер таблицы и название разделены тире, выровнены по левому краю, при необходимости на нескольких строках; 1 --- подпись таблицы не по ГОСТ, на двух и более строках, дальнейшие настройки:
%Выравнивание первой строки, с подписью и номером
\setcounter{tablaba}{2}             % 0 --- по левому краю; 1 --- по центру; 2 --- по правому краю
%Выравнивание строк с самим названием таблицы
\setcounter{tabtita}{1}             % 0 --- по левому краю; 1 --- по центру; 2 --- по правому краю

%%% Рисунки %%%
\DeclareCaptionLabelSeparator*{emdash}{~--- }             % (ГОСТ 2.105, 4.3.1)
\captionsetup[figure]{labelsep=emdash,font=onehalfspacing,position=bottom}

%%% Таблицы %%%
\ifthenelse{\equal{\thetabcap}{0}}{%
    \newcommand{\tabcapalign}{\raggedright}  % по левому краю страницы или аналога parbox
}

\ifthenelse{\equal{\thetablaba}{0} \AND \equal{\thetabcap}{1}}{%
    \newcommand{\tabcapalign}{\raggedright}  % по левому краю страницы или аналога parbox
}

\ifthenelse{\equal{\thetablaba}{1} \AND \equal{\thetabcap}{1}}{%
    \newcommand{\tabcapalign}{\centering}    % по центру страницы или аналога parbox
}

\ifthenelse{\equal{\thetablaba}{2} \AND \equal{\thetabcap}{1}}{%
    \newcommand{\tabcapalign}{\raggedleft}   % по правому краю страницы или аналога parbox
}

\ifthenelse{\equal{\thetabtita}{0} \AND \equal{\thetabcap}{1}}{%
    \newcommand{\tabtitalign}{\raggedright}  % по левому краю страницы или аналога parbox
}

\ifthenelse{\equal{\thetabtita}{1} \AND \equal{\thetabcap}{1}}{%
    \newcommand{\tabtitalign}{\centering}    % по центру страницы или аналога parbox
}

\ifthenelse{\equal{\thetabtita}{2} \AND \equal{\thetabcap}{1}}{%
    \newcommand{\tabtitalign}{\raggedleft}   % по правому краю страницы или аналога parbox
}

\DeclareCaptionFormat{tablenocaption}{\tabcapalign #1\strut}        % Наименование таблицы отсутствует
\ifthenelse{\equal{\thetabcap}{0}}{%
    \DeclareCaptionFormat{tablecaption}{\tabcapalign #1#2#3}
    \captionsetup[table]{labelsep=emdash}                       % тире как разделитель идентификатора с номером от наименования
}{%
    \DeclareCaptionFormat{tablecaption}{\tabcapalign #1#2\par%  % Идентификатор таблицы на отдельной строке
        \tabtitalign{#3}}                                       % Наименование таблицы строкой ниже
    \captionsetup[table]{labelsep=space}                        % пробельный разделитель идентификатора с номером от наименования
}
\captionsetup[table]{format=tablecaption,singlelinecheck=off,font=onehalfspacing,position=top,skip=-5pt}  % многострочные наименования и прочее
\DeclareCaptionLabelFormat{continued}{Продолжение таблицы~#2}
\setlength{\belowcaptionskip}{.2cm}
\setlength{\intextsep}{0ex}

%%% Подписи подрисунков %%%
\renewcommand{\thesubfigure}{\asbuk{subfigure}}           % Буквенные номера подрисунков
\captionsetup[subfigure]{font={normalsize},               % Шрифт подписи названий подрисунков (не отличается от основного)
    labelformat=brace,                                    % Формат обозначения подрисунка
    justification=centering,                              % Выключка подписей (форматирование), один из вариантов
}
%\DeclareCaptionFont{font12pt}{\fontsize{12pt}{13pt}\selectfont} % объявляем шрифт 12pt для использования в подписях, тут же надо интерлиньяж объявлять, если не наследуется
%\captionsetup[subfigure]{font={font12pt}}                 % Шрифт подписи названий подрисунков (всегда 12pt)

%%% Настройки гиперссылок %%%

\definecolor{linkcolor}{rgb}{0.0,0,0}
\definecolor{citecolor}{rgb}{0,0.0,0}
\definecolor{urlcolor}{rgb}{0,0,0}

\hypersetup{
    linktocpage=true,           % ссылки с номера страницы в оглавлении, списке таблиц и списке рисунков
%    linktoc=all,                % both the section and page part are links
%    pdfpagelabels=false,        % set PDF page labels (true|false)
    plainpages=true,           % Forces page anchors to be named by the Arabic form  of the page number, rather than the formatted form
    colorlinks,                 % ссылки отображаются раскрашенным текстом, а не раскрашенным прямоугольником, вокруг текста
    linkcolor={linkcolor},      % цвет ссылок типа ref, eqref и подобных
    citecolor={citecolor},      % цвет ссылок-цитат
    urlcolor={urlcolor},        % цвет гиперссылок
    pdflang={ru},
}
\urlstyle{same}
%%% Шаблон %%%
%\DeclareRobustCommand{\todo}{\textcolor{red}}       % решаем проблему превращения названия цвета в результате \MakeUppercase, http://tex.stackexchange.com/a/187930/79756 , \DeclareRobustCommand protects \todo from expanding inside \MakeUppercase
\setlength{\parindent}{2.5em}                       % Абзацный отступ. Должен быть одинаковым по всему тексту и равен пяти знакам (ГОСТ Р 7.0.11-2011, 5.3.7).

%%% Списки %%%
% Используем дефис для ненумерованных списков (ГОСТ 2.105-95, 4.1.7)
%\renewcommand{\labelitemi}{\normalfont\bfseries~{---}}
\renewcommand{\labelitemi}{\bfseries~{---}}
\setlist{nosep,%                                    % Единый стиль для всех списков (пакет enumitem), без дополнительных интервалов.
    labelindent=\parindent,leftmargin=*%            % Каждый пункт, подпункт и перечисление записывают с абзацного отступа (ГОСТ 2.105-95, 4.1.8)
}
%%%%%%%%%%%%%%%%%%%%%%
%\usepackage{xltxtra} % load xunicode

\usepackage{ragged2e}
\usepackage[explicit]{titlesec}
\usepackage{placeins}
\usepackage{xparse}
\usepackage{csquotes}

\usepackage{listingsutf8}
\usepackage{url} %пакеты расширений
\usepackage{algorithm, algorithmicx}
\usepackage[noend]{algpseudocode}
\usepackage{blkarray}
\usepackage{chngcntr}
\usepackage{tabularx}
\usepackage[backend=biber,
    bibstyle=gost-numeric,
    citestyle=nature]{biblatex}
\newcommand*\template[1]{\text{<}#1\text{>}}

\titleformat{name=\section,numberless}[block]{\normalfont\Large\centering}{}{0em}{#1}
\titleformat{\section}[block]{\normalfont\Large\bfseries\raggedright}{}{0em}{\thesection\hspace{0.25em}#1}
\titleformat{\subsection}[block]{\normalfont\Large\bfseries\raggedright}{}{0em}{\thesubsection\hspace{0.25em}#1}
\titleformat{\subsubsection}[block]{\normalfont\large\bfseries\raggedright}{}{0em}{\thesubsubsection\hspace{0.25em}#1}

\let\Algorithm\algorithm
\renewcommand\algorithm[1][]{\Algorithm[#1]\setstretch{1.5}}
%\renewcommand{\listingscaption}{Листинг}

\usepackage{pifont}
\usepackage{calc}
\usepackage{suffix}
\usepackage{csquotes}
\DeclareQuoteStyle{russian}
    {\guillemotleft}{\guillemotright}[0.025em]
    {\quotedblbase}{\textquotedblleft}
\ExecuteQuoteOptions{style=russian}
\newcommand{\enq}[1]{\enquote{#1}}
\newcommand{\eng}[1]{\begin{english}#1\end{english}}
% Подчиненные счетчики в окружениях http://old.kpfu.ru/journals/izv_vuz/arch/sample1251.tex
\newcounter{cTheorem}
\newcounter{cDefinition}
\newcounter{cConsequent}
\newcounter{cExample}
\newcounter{cLemma}
\newcounter{cConjecture}
\newtheorem{Theorem}{Теорема}[cTheorem]
\newtheorem{Definition}{Определение}[cDefinition]
\newtheorem{Consequent}{Следствие}[cConsequent]
\newtheorem{Example}{Пример}[cExample]
\newtheorem{Lemma}{Лемма}[cLemma]
\newtheorem{Conjecture}{Гипотеза}[cConjecture]

\renewcommand{\theTheorem}{\arabic{Theorem}}
\renewcommand{\theDefinition}{\arabic{Definition}}
\renewcommand{\theConsequent}{\arabic{Consequent}}
\renewcommand{\theExample}{\arabic{Example}}
\renewcommand{\theLemma}{\arabic{Lemma}}
\renewcommand{\theConjecture}{\arabic{Conjecture}}
%\makeatletter
\NewDocumentCommand{\Newline}{}{\text{\\}}
\newcommand{\sequence}[2]{\ensuremath \left(#1,\ \dots,\ #2\right)}

\definecolor{mygreen}{rgb}{0,0.6,0}
\definecolor{mygray}{rgb}{0.5,0.5,0.5}
\definecolor{mymauve}{rgb}{0.58,0,0.82}
\renewcommand{\listalgorithmname}{Список алгоритмов}
\floatname{algorithm}{Листинг}
\renewcommand{\lstlistingname}{Листинг}
\renewcommand{\thealgorithm}{\arabic{algorithm}}

\newcommand{\refAlgo}[1]{(листинг \ref{#1})}
\newcommand{\refImage}[1]{(рисунок \ref{#1})}

\renewcommand{\theenumi}{\arabic{enumi}.}% Меняем везде перечисления на цифра.цифра
\renewcommand{\labelenumi}{\arabic{enumi}.}% Меняем везде перечисления на цифра.цифра
\renewcommand{\theenumii}{\arabic{enumii}}% Меняем везде перечисления на цифра.цифра
\renewcommand{\labelenumii}{(\arabic{enumii})}% Меняем везде перечисления на цифра.цифра
\renewcommand{\theenumiii}{\roman{enumiii}}% Меняем везде перечисления на цифра.цифра
\renewcommand{\labelenumiii}{(\roman{enumiii})}% Меняем везде перечисления на цифра.цифра
\renewcommand{\labelitemi}{---}
\renewcommand{\labelitemii}{---}

\makeatletter
\def\p@subsection{}
\def\p@subsubsection{\thesection\,\thesubsection\,}
\makeatother
\newcommand{\anonsection}[1]{\cleardoublepage
\phantomsection
\addcontentsline{toc}{section}{\protect\numberline{}#1}
\section*{#1}\vspace*{2.5ex} % По госту положены 3 пустые строки после заголовка ненумеруемого раздела
}
\newcommand{\sectionbreak}{\clearpage}
\renewcommand{\sectionfont}{\normalsize} % Сбиваем стиль оглавления в стандартный
\renewcommand{\cftsecleader}{\cftdotfill{\cftdotsep}} % Точки в оглавлении напротив разделов

\renewcommand{\cftsecfont}{\normalfont\large} % Переключение на times в содержании
\renewcommand{\cftsubsecfont}{\normalfont\large} % Переключение на times в содержании

\setlength{\cftsecindent}{0pt}% Убираем отступы в содержании для \section
\setlength{\cftsubsecindent}{0pt}% Убираем отступы в содержании для \subsection
\setlength{\cftsubsubsecindent}{0pt}% Убираем отступы в содержании для \subsubsection

\usepackage{caption}
%\captionsetup[table]{justification=raggedleft}
%\captionsetup[figure]{justification=centering,labelsep=endash}
\usepackage{amsmath}    % \bar    (матрицы и проч. ...)
\usepackage{amsfonts}   % \mathbb (символ для множества действительных чисел и проч. ...)
\usepackage{mathtools}  % \abs, \norm
    \DeclarePairedDelimiter\abs{\lvert}{\rvert} % операция модуля
    \DeclarePairedDelimiter\norm{\lVert}{\rVert} % операция нормы
\DeclareTextCommandDefault{\textvisiblespace}{%
  \mbox{\kern.06em\vrule \@height.3ex}%
  \vbox{\hrule \@width.3em}%
  \hbox{\vrule \@height.3ex}}
\newsavebox{\spacebox}
\begin{lrbox}{\spacebox}
\verb*! !
\end{lrbox}
\newcommand{\aspace}{\usebox{\spacebox}}
\DeclareTotalCounter{listing}
\def\figurename{Рисунок}

\makeatletter
\renewcommand*{\p@subsubsection}{}
\makeatother

\newcommand{\bmstuheader}
{\vspace*{-30pt}
\hspace{-45pt}
\begin{minipage}{0.17\textwidth}
\hspace*{-20pt}\centering
\includegraphics[width=1.3\textwidth]{emblem.png}
\end{minipage}
\begin{minipage}{0.82\textwidth}\small \textbf{
\vspace*{-0.7ex}
\hspace*{-10pt}\centerline{Министерство науки и высшего образования Российской Федерации}
\vspace*{-0.7ex}
\centerline{Федеральное государственное автономное образовательное учреждение }
\vspace*{-0.7ex}
\centerline{высшего образования}
\vspace*{-0.7ex}
\centerline{<<Московский государственный технический университет}
\vspace*{-0.7ex}
\centerline{имени Н.Э. Баумана}
\vspace*{-0.7ex}
\centerline{(национальный исследовательский университет)>>}
\vspace*{-0.7ex}
\centerline{(МГТУ им. Н.Э. Баумана)}}
\end{minipage}

\vspace{-2pt}
\hspace{-34.5pt}\rule{\textwidth}{0.5pt}

\vspace*{-18.3pt}
\hspace{-34.5pt}\rule{\textwidth}{2.5pt}

\vspace{0.5ex}
\noindent \small ФАКУЛЬТЕТ\hspace{80pt} <<Информатика и системы управления>>

\vspace*{-16pt}
\hspace{35pt}\rule{0.855\textwidth}{0.4pt}

\vspace{0.5ex}
\noindent \small КАФЕДРА\hspace{50pt} <<Теоретическая информатика и компьютерные технологии>>

\vspace*{-16pt}
\hspace{25pt}\rule{0.875\textwidth}{0.4pt}
}

\addbibresource{biblio.bib}

\usepackage{libertine}

\DeclareFieldFormat{url}{%
  \mkbibacro{URL}\addcolon\space
  \href{#1}{\nolinkurl{\thefield{urlraw}}}}

\usepackage{tikz}
\usetikzlibrary{shapes.geometric}

\begin{document}
\sloppy

\begin{titlepage}
\thispagestyle{empty}

\bmstuheader

\vspace{3em}

\begin{center}
\Large {\textbf{\uline{ОТЧЁТ ПО ПРОИЗВОДСТВЕННОЙ}}} \\ {\textbf{\uline{ПРАКТИКЕ}}}
\end{center}\normalsize

\vspace{2ex}
\noindent Студент \underline{\hspace{406pt}}

\vspace{-2.1ex}
\noindent\hspace{31ex}\scriptsize{(Фамилия, Имя, Отчество)}\normalsize

\vspace{0.7ex}
\noindent Группа \underline{\hspace{50pt}}

\vspace{2ex}
\noindent Тип практики \underline{\hspace{373pt}}

\vspace{2.5ex}
\noindent Название предприятия \underline{\hspace{16pt}\footnotesize Институт Программных Систем им. А.К. Айламазяна РАН\hspace{12pt}}\normalsize

\vspace{\fill}


\noindent Студент \underline{\hspace{1.5cm}} \hfill \underline{\hspace{4cm}}\quad
\underline{\hspace{4cm}}

\vspace{-2.1ex}
\noindent\hspace{9ex}\scriptsize{(Группа)}\normalsize\hspace{170pt}\hspace{2ex}\scriptsize{(Подпись, дата)}\normalsize\hspace{30pt}\hspace{6ex}\scriptsize{(И.О. Фамилия)}\normalsize

\bigskip

\noindent Рекомендуемая оценка\hfill\underline{\hspace{195pt}}\hfill

\bigskip

\noindent \parbox{0.333\textwidth}{Руководитель практики\\от предприятия}
\hfill \underline{\hspace{4cm}}\quad \underline{\hspace{4cm}}

\vspace{-2ex}
\noindent\hspace{13.5ex}\normalsize\hspace{170pt}\hspace{2ex}\scriptsize{(Подпись, дата)}\normalsize\hspace{30pt}\hspace{6ex}\scriptsize{(И.О. Фамилия)}\normalsize


\bigskip

\noindent Руководитель практики  \hfill \underline{\hspace{4cm}}\quad
\underline{\hspace{4cm}}

\vspace{-2ex}
\noindent\hspace{13.5ex}\normalsize\hspace{170pt}\hspace{2ex}\scriptsize{(Подпись, дата)}\normalsize\hspace{30pt}\hspace{6ex}\scriptsize{(И.О. Фамилия)}\normalsize

\vspace{2ex}
\noindent Оценка\hfill\underline{\hspace{195pt}}

\vfill

%\vspace{\fill}



\begin{center}
\textsl{\the\year{} г.}
\end{center}
\end{titlepage}

\setlength{\tabcolsep}{3pt}
\newpage
\setcounter{page}{2}
%----------------------------------------------------------------------------
%                  ОТСЮДА --- СОБСТВЕННО ТЕКСТ
%----------------------------------------------------------------------------
\renewcommand\contentsname{\hfill{\normalfont{СОДЕРЖАНИЕ}}\hfill}  %Оглавление
\tableofcontents
\newpage
\section{Характеристика предприятия}

\bigskip

Производственная практика проходила в Институте программных систем
Российской академии наук (кратко --- ИПС им. А.К. Айламазяна РАН). Институт Программных Систем был создан в апреле 1984 года как Филиал Института проблем кибернетики АН СССР по решению Правительства СССР, направленному на развитие вычислительной техники и информатики в стране. Руководителем ФИПК АН СССР был назначен д.т.н., профессор Альфред Карлович Айламазян. В 1986 году Филиал Института проблем кибернетики был преобразован в Институт программных систем АН СССР, а в 2008 году институту было присвоено имя его первого директора профессора А.К. Айламазяна.

С момента создания основными научными направлениями
деятельности института являлись:
\begin{itemize}
\item высокопроизводительные вычисления,
\item программные системы для параллельных архитектур,
\item автоматизация программирования,
\item телекоммуникационные системы и медицинская информатика.
\end{itemize}

Сегодня Институт программных систем имени А.К. Айламазяна РАН объединяет пять исследовательских центров:
\begin{itemize}
\item Исследовательский центр мультипроцессорных систем (ИЦМС),
\item Исследовательский центр медицинской информатики (ИЦМИ Интерин),
\item Исследовательский центр искусственного интеллекта (ИЦИИ),
\item Исследовательский центр процессов управления (ИЦПУ),
\item Исследовательский центр системного анализа (ИЦСА).
\end{itemize}

Практика проходила в исследовательском центре мультипроцессорных систем, в лаборатории автоматизации программирования.

\newpage
\section{Индивидуальное задание}

Конечный моноид (моноид трансформаций) над алфавитом $\Sigma$ определяется
конечным множеством слов $M$ и таблицей трансформаций, соответствующей действиям
букв на элементы $M$.

Множество элементов моноида $I$ является двухсторонним (левым, правым) идеалом,
если $MIM = I (MI = I, IM = I)$.

По моноиду трансформаций построить множества его идеалов.

\newpage
\section{Теоретическая часть}

\begin{Definition}[Конечный моноид]
  $(M, \cdot, 1)$ --- конечный моноид, если:
  \begin{enumerate}
    \item M - конечное множество;
    \item $\cdot$ - замкнутая, ассоциативная, бинарная операция над $M$;
    \item $\forall x \in M: 1 \cdot x = x \cdot 1 = x$.
  \end{enumerate}
\end{Definition}

\begin{Definition}[Трансформация]
  Трансформация множества $X$ --- отображение $X \mapsto X$.
\end{Definition}

\begin{Definition}[Моноид трансформаций]
  Пусть дано множество $M$. Множество трансформаций над $M$, замкнутое
  относительно операции $\circ$ (композиция), называется моноидом трансформаций
  над $M$. При этом единицей моноида трансформаций является эквивалентная
  трансформация.
\end{Definition}

\begin{Definition}[Идеал]
  $I \subset M$ --- идеал моноида $M$, если $MIM = I$. Т.е.
  $\forall x, y \in M, k \in I: xky \in I$.
\end{Definition}

\begin{Definition}[Левый идеал]
  $I \subset M$ --- левый идеал моноида $M$, если $MI = I$. Т.е.
  $\forall x \in M, k \in I: xk \in I$.
\end{Definition}

\begin{Definition}[Правый идеал]
  $I \subset M$ --- правый идеал моноида $M$, если $IM = I$. Т.е.
  $\forall x \in M, k \in I: kx \in I$.
\end{Definition}

\begin{Definition}[Алфавит]
  Алфавит --- конечное множество букв.
\end{Definition}

\begin{Example}[Алфавит]
  $\Sigma = \{a, b, c\}$
\end{Example}

\begin{Definition}[Отношения Грина]
  \begin{equation}
    s \leq_{\mathcal{R}} t \Leftrightarrow \exists u \in S^{1} : s = tu
  \end{equation}
  \begin{equation}
    s \leq_{\mathcal{L}} t \Leftrightarrow \exists u \in S^{1} : s = ut
  \end{equation}
  \begin{equation}
    s \leq_{\mathcal{J}} t \Leftrightarrow \exists u, v \in S^{1} : s = utv
  \end{equation}
  \begin{equation}
    s \leq_{\mathcal{H}} t \Leftrightarrow s \leq_{\mathcal{R}} t \quad{}\text{и}\quad{} s \leq_{\mathcal{L}} t
  \end{equation}
  Через идеалы:
  \begin{equation}
    s \leq_{\mathcal{R}} t \Leftrightarrow s S^{1} \subseteq t S^{1}
  \end{equation}
  \begin{equation}
    s \leq_{\mathcal{L}} t \Leftrightarrow S^{1}s  \subseteq S^{1} t
  \end{equation}
  \begin{equation}
    s \leq_{\mathcal{J}} t \Leftrightarrow S^{1}s S^{1}  \subseteq S^{1} t S^{1}
  \end{equation}
  \begin{equation}
    s \leq_{\mathcal{H}} t \Leftrightarrow s \leq_{\mathcal{R}} t \quad{}\text{и}\quad{} s \leq_{\mathcal{L}} t
  \end{equation}
  где $sS^{1}$ --- правый идеал, порожденный элементом $s$ (т.е. минимальный
  идеал, содержащий $s$).
\end{Definition}

Таблица трансформаций задает действие буквы на элементы множества $M$ (слова).
То есть эта таблица задает соответствие между буквами и трансформациями
множества $M$.

Множество элементов моноида трансформаций можно построить с помощью замыкания по
операции композиции и представить в виде графа Кэли, где вершины соответствуют
трансформациям, а ребра соединяют вершины, если конечная вершина ребра может
быть получена с помощью композиции транформаций соответствующих начальной
вершины ребра и букве, которой помечают ребро.

При построении левого графа Кэли производится композиция транформации
соответствущей букве и трансформации соответствующей вершине, а в правом графе
Кэли наоборот: вершина к букве.

\begin{Example}\label{example:simple}
  Для $M = \{x, y, z\}$, и таблицы трансформаций:

  \begin{table}[htb]
    \centering\begin{tabular}{c|ccc}
      & a & b & c \\
      \hline
      x & x & z & z \\
      y & y & y & z \\
      z & x & x & z
    \end{tabular}
  \end{table}

  Правый граф Кэли представлен на рисунке~\ref{fig:example1}, левый граф Кэли
procedure  представлен на рисунке~\ref{fig:example_left}.

  \begin{figure}[!htb]
    \centering
    \begin{minipage}[t]{\textwidth}
      \centering
      
\begin{tikzpicture}[>=latex,line join=bevel,]
%%
\node (a) at (69.0bp,285.75bp) [draw,ellipse] {a};
  \node (c) at (104.0bp,107.25bp) [draw,ellipse] {c};
  \node (ab) at (57.0bp,196.5bp) [draw,ellipse] {ab};
  \node (b) at (71.0bp,464.25bp) [draw,ellipse] {b};
  \node (bb) at (114.0bp,375.0bp) [draw,ellipse] {bb};
  \node (ca) at (104.0bp,18.0bp) [draw,ellipse] {ca};
  \draw [->] (a) ..controls (104.51bp,294.77bp) and (114.0bp,292.03bp)  .. (114.0bp,285.75bp) .. controls (114.0bp,281.83bp) and (110.29bp,279.28bp)  .. (a);
  \definecolor{strokecol}{rgb}{0.0,0.0,0.0};
  \pgfsetstrokecolor{strokecol}
  \draw (118.12bp,285.75bp) node {a};
  \draw [->] (a) ..controls (82.457bp,263.09bp) and (85.831bp,256.31bp)  .. (88.0bp,249.75bp) .. controls (100.42bp,212.16bp) and (103.56bp,166.24bp)  .. (c);
  \draw (105.39bp,196.5bp) node {c};
  \draw [->] (a) ..controls (75.325bp,257.34bp) and (76.655bp,244.04bp)  .. (74.0bp,232.5bp) .. controls (73.351bp,229.68bp) and (72.442bp,226.82bp)  .. (ab);
  \draw (79.988bp,241.12bp) node {b};
  \draw [->] (b) ..controls (70.357bp,431.71bp) and (69.974bp,411.05bp)  .. (69.75bp,393.0bp) .. controls (69.424bp,366.79bp) and (69.229bp,336.94bp)  .. (a);
  \draw (73.875bp,375.0bp) node {a};
  \draw [->] (b) ..controls (30.942bp,436.46bp) and (0.0bp,409.0bp)  .. (0.0bp,376.0bp) .. controls (0.0bp,376.0bp) and (0.0bp,376.0bp)  .. (0.0bp,195.5bp) .. controls (0.0bp,158.29bp) and (39.558bp,133.79bp)  .. (c);
  \draw (3.75bp,285.75bp) node {c};
  \draw [->] (b) ..controls (78.592bp,435.8bp) and (83.197bp,422.26bp)  .. (89.0bp,411.0bp) .. controls (90.8bp,407.51bp) and (92.915bp,403.98bp)  .. (bb);
  \draw (93.5bp,419.62bp) node {b};
  \draw [->] (c) ..controls (139.51bp,116.27bp) and (149.0bp,113.53bp)  .. (149.0bp,107.25bp) .. controls (149.0bp,103.33bp) and (145.29bp,100.78bp)  .. (c);
  \draw (152.75bp,107.25bp) node {c};
  \draw [->] (c) ..controls (73.592bp,89.748bp) and (63.826bp,81.592bp)  .. (58.75bp,71.25bp) .. controls (55.372bp,64.367bp) and (55.372bp,60.883bp)  .. (58.75bp,54.0bp) .. controls (62.16bp,47.051bp) and (67.688bp,41.089bp)  .. (ca);
  \draw (62.875bp,62.625bp) node {a};
  \draw [->] (c) ..controls (92.806bp,84.213bp) and (90.303bp,77.596bp)  .. (89.0bp,71.25bp) .. controls (87.458bp,63.74bp) and (87.458bp,61.51bp)  .. (89.0bp,54.0bp) .. controls (89.55bp,51.323bp) and (90.313bp,48.597bp)  .. (ca);
  \draw (93.5bp,62.625bp) node {b};
  \draw [->] (ab) ..controls (36.478bp,222.95bp) and (29.611bp,237.15bp)  .. (34.75bp,249.75bp) .. controls (36.809bp,254.8bp) and (39.985bp,259.51bp)  .. (a);
  \draw (38.875bp,241.12bp) node {a};
  \draw [->] (ab) ..controls (60.992bp,226.52bp) and (63.189bp,242.5bp)  .. (a);
  \draw (68.575bp,241.12bp) node {b};
  \draw [->] (ab) ..controls (72.602bp,166.54bp) and (82.06bp,148.98bp)  .. (c);
  \draw (88.46bp,151.88bp) node {c};
  \draw [->] (bb) ..controls (99.062bp,345.04bp) and (90.007bp,327.48bp)  .. (a);
  \draw (99.656bp,330.38bp) node {a};
  \draw [->] (bb) ..controls (107.82bp,403.58bp) and (103.75bp,417.13bp)  .. (98.0bp,428.25bp) .. controls (96.053bp,432.02bp) and (93.689bp,435.79bp)  .. (b);
  \draw (109.46bp,419.62bp) node {b};
  \draw [->] (bb) ..controls (123.46bp,342.79bp) and (128.8bp,322.19bp)  .. (131.0bp,303.75bp) .. controls (136.91bp,254.23bp) and (119.54bp,149.49bp)  .. (118.0bp,143.25bp) .. controls (117.35bp,140.62bp) and (116.55bp,137.93bp)  .. (c);
  \draw (134.84bp,241.12bp) node {c};
  \draw [->] (ca) ..controls (104.0bp,48.024bp) and (104.0bp,64.0bp)  .. (c);
  \draw (108.5bp,62.625bp) node {b};
  \draw [->] (ca) ..controls (113.8bp,41.422bp) and (115.9bp,47.875bp)  .. (117.0bp,54.0bp) .. controls (118.36bp,61.546bp) and (118.36bp,63.704bp)  .. (117.0bp,71.25bp) .. controls (116.55bp,73.738bp) and (115.94bp,76.28bp)  .. (c);
  \draw (121.77bp,62.625bp) node {c};
  \draw [->] (ca) ..controls (139.51bp,27.021bp) and (149.0bp,24.275bp)  .. (149.0bp,18.0bp) .. controls (149.0bp,14.078bp) and (145.29bp,11.535bp)  .. (ca);
  \draw (153.12bp,18.0bp) node {a};
%
\end{tikzpicture}


    \end{minipage}
    \caption{Правый граф Кэли для примера~\ref{example:simple}.}
    \label{fig:example1}
  \end{figure}

  \begin{figure}[!htb]
    \centering
    \begin{minipage}[t]{\textwidth}
      \centering
      
\begin{tikzpicture}[>=latex,line join=bevel,]
%%
\node (a) at (217.0bp,107.25bp) [draw,ellipse] {a};
  \node (ca) at (27.0bp,18.0bp) [draw,ellipse] {ca};
  \node (b) at (105.0bp,285.75bp) [draw,ellipse] {b};
  \node (ab) at (66.0bp,196.5bp) [draw,ellipse] {ab};
  \node (bb) at (191.0bp,196.5bp) [draw,ellipse] {bb};
  \node (c) at (66.0bp,107.25bp) [draw,ellipse] {c};
  \draw [->] (a) ..controls (253.53bp,111.6bp) and (262.0bp,110.21bp)  .. (262.0bp,107.25bp) .. controls (262.0bp,105.54bp) and (259.17bp,104.35bp)  .. (a);
  \definecolor{strokecol}{rgb}{0.0,0.0,0.0};
  \pgfsetstrokecolor{strokecol}
  \draw (266.12bp,107.25bp) node {a};
  \draw [->] (a) ..controls (260.82bp,117.23bp) and (280.0bp,114.79bp)  .. (280.0bp,107.25bp) .. controls (280.0bp,101.24bp) and (267.82bp,98.47bp)  .. (a);
  \draw (284.5bp,107.25bp) node {b};
  \draw [->] (a) ..controls (190.6bp,93.732bp) and (185.64bp,91.403bp)  .. (181.0bp,89.25bp) .. controls (139.04bp,69.771bp) and (90.441bp,47.681bp)  .. (ca);
  \draw (142.53bp,62.625bp) node {c};
  \draw [->] (b) ..controls (67.304bp,260.99bp) and (41.409bp,240.28bp)  .. (30.0bp,214.5bp) .. controls (5.5951bp,159.37bp) and (13.612bp,86.622bp)  .. (ca);
  \draw (19.401bp,151.88bp) node {c};
  \draw [->] (b) ..controls (92.143bp,255.99bp) and (84.456bp,238.79bp)  .. (ab);
  \draw (93.118bp,241.12bp) node {a};
  \draw [->] (b) ..controls (126.99bp,259.07bp) and (139.91bp,244.57bp)  .. (152.0bp,232.5bp) .. controls (156.59bp,227.92bp) and (161.64bp,223.18bp)  .. (bb);
  \draw (156.5bp,241.12bp) node {b};
  \draw [->] (c) ..controls (102.78bp,110.12bp) and (111.0bp,109.2bp)  .. (111.0bp,107.25bp) .. controls (111.0bp,106.12bp) and (108.25bp,105.34bp)  .. (c);
  \draw (115.12bp,107.25bp) node {a};
  \draw [->] (c) ..controls (110.7bp,113.83bp) and (129.0bp,112.16bp)  .. (129.0bp,107.25bp) .. controls (129.0bp,103.34bp) and (117.38bp,101.48bp)  .. (c);
  \draw (133.5bp,107.25bp) node {b};
  \draw [->] (c) ..controls (117.88bp,117.62bp) and (147.0bp,115.44bp)  .. (147.0bp,107.25bp) .. controls (147.0bp,100.24bp) and (125.69bp,97.636bp)  .. (c);
  \draw (150.75bp,107.25bp) node {c};
  \draw [->] (ca) ..controls (63.78bp,20.874bp) and (72.0bp,19.947bp)  .. (72.0bp,18.0bp) .. controls (72.0bp,16.875bp) and (69.253bp,16.09bp)  .. (ca);
  \draw (76.125bp,18.0bp) node {a};
  \draw [->] (ca) ..controls (71.696bp,24.579bp) and (90.0bp,22.91bp)  .. (90.0bp,18.0bp) .. controls (90.0bp,14.087bp) and (78.377bp,12.233bp)  .. (ca);
  \draw (94.5bp,18.0bp) node {b};
  \draw [->] (ca) ..controls (78.877bp,28.369bp) and (108.0bp,26.193bp)  .. (108.0bp,18.0bp) .. controls (108.0bp,10.991bp) and (86.687bp,8.3856bp)  .. (ca);
  \draw (111.75bp,18.0bp) node {c};
  \draw [->] (ab) ..controls (66.0bp,166.43bp) and (66.0bp,150.45bp)  .. (c);
  \draw (69.75bp,151.88bp) node {c};
  \draw [->] (ab) ..controls (102.53bp,200.85bp) and (111.0bp,199.46bp)  .. (111.0bp,196.5bp) .. controls (111.0bp,194.79bp) and (108.17bp,193.6bp)  .. (ab);
  \draw (115.12bp,196.5bp) node {a};
  \draw [->] (ab) ..controls (109.82bp,206.48bp) and (129.0bp,204.04bp)  .. (129.0bp,196.5bp) .. controls (129.0bp,190.49bp) and (116.82bp,187.72bp)  .. (ab);
  \draw (133.5bp,196.5bp) node {b};
  \draw [->] (bb) ..controls (199.59bp,166.68bp) and (204.48bp,150.28bp)  .. (a);
  \draw (210.45bp,151.88bp) node {a};
  \draw [->] (bb) ..controls (178.95bp,224.98bp) and (171.01bp,239.35bp)  .. (161.0bp,249.75bp) .. controls (153.97bp,257.05bp) and (145.11bp,263.48bp)  .. (b);
  \draw (178.46bp,241.12bp) node {b};
  \draw [->] (bb) ..controls (151.99bp,168.27bp) and (117.63bp,144.29bp)  .. (c);
  \draw (143.45bp,151.88bp) node {c};
%
\end{tikzpicture}


    \end{minipage}
    \caption{Левый граф Кэли для примера~\ref{example:simple}.}
    \label{fig:example_left}
  \end{figure}
\end{Example}

% NOTE: можно добавить псевдокод алгоритма нахождения элементов

Так как в моноиде все элементы образуются с помощью композиции образующих, из
определения идеала следует, что применение буквенной трансформации к элементу
идеала должно давать в результате также элемент идеала. В терминологии графов
Кэли это означает, что из вершин левого идеала в левом графе Кэли не должно быть
ребер в вершины не принадлежащие искомому идеалу. Для вершин из правого идеала
должно выполняться аналогичное условие в правом графе Кэли.

Таким образом, идеалами являются сочетания компонент связности графа Кэли, из
вершин которых нет исходящих ребер. То есть для любой компоненты связности из
идеала верно, что все достижимые из нее компоненты также принадлежат идеалу.

Интересно, что отношения Грина связаны с графами Кэли: $\mathcal{R}$-классы
соответствуют компонентам связности в правом графе Кэли, а $\mathcal{L}$-классы
--- в левом~\cite{mathfoundations2025}.

Для построения множества идеалов необходимо найти конденсацию графа. Это
делается с помощью двух запусков обхода в глубину~\refAlgo{lst:condensation}~\cite{sccs}.

Построенный граф конденсации является направленным и
ацикличным~\refImage{fig:condensation}. В таком графе нужно найти все сочетания
компонент связности, для которых выполняется условие, что для каждой компоненты
сильной связности в сочетание входят все достижимые из нее компоненты. Множества
вершин для каждого сочетания образуют множество искомых идеалов соответствующего
моноида трансформаций.

\begin{figure}[!htb]
  \centering
  \begin{minipage}[t]{\textwidth}
    \centering
    
\begin{tikzpicture}[>=latex,line join=bevel,]
%%
\node (b+ bb) at (70.425bp,196.5bp) [draw,ellipse] {b, bb};
  \node (a+ ab) at (35.425bp,107.25bp) [draw,ellipse] {a, ab};
  \node (c+ ca) at (70.425bp,18.0bp) [draw,ellipse] {c, ca};
  \draw [->] (b+ bb) ..controls (58.817bp,166.56bp) and (52.153bp,149.95bp)  .. (a+ ab);
  \definecolor{strokecol}{rgb}{0.0,0.0,0.0};
  \pgfsetstrokecolor{strokecol}
  \draw (60.185bp,151.88bp) node {a};
  \draw [->] (b+ bb) ..controls (75.439bp,164.05bp) and (78.266bp,143.41bp)  .. (79.425bp,125.25bp) .. controls (80.443bp,109.28bp) and (80.443bp,105.22bp)  .. (79.425bp,89.25bp) .. controls (78.547bp,75.487bp) and (76.711bp,60.3bp)  .. (c+ ca);
  \draw (83.939bp,107.25bp) node {c};
  \draw [->] (a+ ab) ..controls (47.032bp,77.314bp) and (53.697bp,60.701bp)  .. (c+ ca);
  \draw (59.81bp,62.625bp) node {c};
%
\end{tikzpicture}


  \end{minipage}
  \caption{Граф конденсации для правого графа Кэли.}
  \label{fig:condensation}
\end{figure}

Такие сочетания можно найти пронумеровав вершины графа конденсации и рекурсивно
сгенерировав все сочетания этих вершин от одного до $n = |G|$
элементов~\cite{combobjects}. Причем, нужно исключить сочетания, в которых есть
пара вершин, такая, что одна достижима из другой. После этого для каждого
сочетания находим все достижимые из его элементов вершины, в резльтате получая
идеалы. Алгоритм нахождения левых (правых) идеалов по графу конденсации
представлен на листинге~\ref{lst:ideals}.

\begin{listing}[!htb]
\caption{Алгоритм нахождения идеалов по графу конденсации.}
\label{lst:ideals}
\begin{algorithmic}[1]

\State global Condensation(V, E)

\Procedure{Ideals}{Condensation(V, E)}
    \State $\text{prefix} \gets \{\}$
    \State $\text{ideals} \gets \{\}$

    \State $\text{Generate(prefix, ideals)}$
    \State \Return $\text{ideals}$
\EndProcedure

\Function{Generate}{prefix, ideals}
    \State $\text{components} \gets \text{(descendants of vertices in prefix)}$
    \State $\text{ideal} \gets \{\text{vertices of component}\quad{} | \quad{} \text{for component} \in \text{components}\}$
    \State $\text{ideals} \gets \text{ideal}$
    \If{$|\text{prefix}| = |V|$}
      \State \Return
    \EndIf
    \State $\text{from} \gets |\text{prefix}|$
    \For{$i \in [\text{from}, |\text{V}|]$}
        \State $\text{prefix.append(i)}$
        \State $\text{Generate(prefix, ideals)}$
        \State $\text{prefix.pop()}$
    \EndFor
\EndFunction
\end{algorithmic}
\end{listing}

После получения множеств левых и правых идеалов, двусторонние идеалы находятся
тривиальным пересечением множеств.

\newpage
\section{Реализация}

Для реализации был выбран язык программирования C++, ввиду удобства его
стандартной библиотеки для написания алгоритмов. Использован современный
стандарт C++23, что позволило пользоваться библиотекой ranges для более
декларативного кода.

Вершина графа хранится как структура, состоящая из слова, трансформации и списка
смежности~\refAlgo{lst:monoid_models}. Трансформация хранится как индексы
элементов множества $M$.

\begin{listing}[!htb]
\caption{Объявление типов для хранения графа Кэли.}
\label{lst:monoid_models}
  \begin{minted}[style=bw, breaklines, frame=single, fontsize = \small, linenos=false, xleftmargin = 1.5em]{cpp}
using Word = std::string;
using Transformation = std::vector<size_t>;
using LetterToTransformation = std::map<char, Transformation>;
using ElementIndex = size_t;
struct MonoidElement {
  std::string word;
  Transformation transformation;
  std::vector<ElementIndex> transitions;

  bool operator<=>(const MonoidElement& other) const = default;
};
using CayleyGraph = std::vector<MonoidElement>;
  \end{minted}
\end{listing}

За построение графов Кэли отвечает класс
\verb|CayleyGraphBuilder|~\refAlgo{lst:graph_builder}. Данный класс строит
вершины графа с помощью замыкания по операции композиции. Для построения левого
или правого графа Кэли предусмотрена передача в конструктор стратегий композиции
и именования вершин, реализованная с помощью \verb|std::function|. Использование
такого паттерна является примером использования семантики значений в языке C++,
что делает код проще и снижает вероятность ошибок.

\begin{listing}[!htb]
\caption{Объявление класса для построения графов Кэли.}
\label{lst:graph_builder}
  \begin{minted}[style=bw, breaklines, frame=single, fontsize = \small, linenos=false, xleftmargin = 1.5em]{cpp}
using CompositionStrategy
    = std::function<Transformation(const Transformation& lhs, const Transformation& rhs)>;
using WordCompositionStrategy
    = std::function<std::string(const MonoidElement& lhs, const MonoidElement& rhs)>;

class CayleyGraphBuilder {
public:
  CayleyGraphBuilder(const LetterToTransformation& letter_transformations,
                     CompositionStrategy composition_strategy = RightComposition,
                     WordCompositionStrategy word_composition_strategy = RightWordComposition);
  CayleyGraph Build();
private:
  void AddComposition(size_t element_index, size_t letter_index);
private:
  CayleyGraph monoid_elements_;
  std::map<Transformation, size_t> monoid_transformations_;
  CompositionStrategy composition_strategy_;
  WordCompositionStrategy word_composition_strategy_;
};
  \end{minted}
\end{listing}

Из реализации построения графа конденсации в отдельный класс выделен алгоритм
топологической сортировки. За нахождение множеств идеалов ответственнен класс
\verb|IdealsBuilder|~\refAlgo{lst:ideals_builder_impl}. Он строит левые или
правые идеалы в зависимости от того, какой граф Кэли был передан в конструктор.
Перебор комбинаторных объектов реализован с помощью рекурсии, а нахождение
множества достижимых из вектора вершин с помощью поиска в ширину.

\begin{listing}[!htb]
\caption{Объявление класса для построения левых или правых идеалов.}
\label{lst:ideals_builder_impl}
  \begin{minted}[style=bw, breaklines, frame=single, fontsize = \small, linenos=false, xleftmargin = 1.5em]{cpp}
class IdealsBuilder {
public:
  IdealsBuilder(CayleyGraph monoid, CondensationGraph graph);
  std::vector<std::vector<ElementIndex>> Build();
private:
  std::vector<size_t> GetDescentants(std::vector<size_t> vertices);
  void Generate(std::vector<size_t>& prefix, std::vector<std::vector<ElementIndex>>& ideals);
private:
  CayleyGraph monoid_;
  CondensationGraph graph_;
  StronglyConnectedComponents scc_;
};
  \end{minted}
\end{listing}

Ввод производится из стандартного потока, но легко поддержать ввод из файла или
строки. Вывод также производится в стандартный поток, а также функцией
\verb|VisualizeDot| записываются файлы c кодом на \verb|dot|, которые содержат
визуализацию левого и правого графов Кэли, что удобно для проверки.

\newpage
\section{Отладка и тестирование}

Модульные тесты реализованы с использованием библиотеки Google Tests.

Покрытие включает визуализацию, все алгоритмы и ввод.

Пример теста, который проверяет построение правого графа Кэли приведен на
листинге~\ref{lst:test_build}. В данном тесте определяется таблица, задающая
трансформации соответствующие буквам. Затем строится правый граф Кэли. После
проверяется количество вершин в графе и трансформации, соответствующие
добавленным вершинам.

\begin{listing}[!htb]
\caption{Тест, проверяющий построение правого графа Кэли.}
\label{lst:test_build}
  \begin{minted}[style=bw, breaklines, frame=single, fontsize = \small, linenos=false, xleftmargin = 1.5em]{cpp}
TEST(MonoidBuilderTest, FindsClosureByTransformationComposition) {
  LetterToTransformation letter_transformations{
      {'a', {0, 1, 0}},
      {'b', {2, 1, 0}},
      {'c', {2, 2, 2}},
  };
  CayleyGraphBuilder builder(std::move(letter_transformations));

  auto got = builder.Build();

  ASSERT_THAT(got.size(), Eq(6));
  ASSERT_THAT(got[3].transformation, Eq(Transformation{2, 1, 2}));
  ASSERT_THAT(got[4].transformation, Eq(Transformation{0, 1, 2}));
  ASSERT_THAT(got[5].transformation, Eq(Transformation{0, 0, 0}));
}
  \end{minted}
\end{listing}

Пример теста, который проверяет построение множество правых идеалов приведен на
листинге~\ref{lst:test_ideals}. В этом тесте также сначала определяется таблица
трансформаций. Затем строится правый граф Кэли и производится его конденсация.
Затем по графу Кэли и графу конденсации находится множество правых идеалов и
сравнивается с ожидаемым.

\begin{listing}[!htb]
\caption{Тест, проверяющий построение множества правых идеалов.}
\label{lst:test_ideals}
  \begin{minted}[style=bw, breaklines, frame=single, fontsize = \small, linenos=false, xleftmargin = 1.5em]{cpp}
TEST(StrongConnectivityTest, FindRightIdeals) {
  LetterToTransformation letter_transformations{
      {'a', {0, 1, 0}},
      {'b', {2, 1, 0}},
      {'c', {2, 2, 2}},
  };
  auto monoid = CayleyGraphBuilder(letter_transformations).Build();
  auto condensation_graph = CondensationGraphBuilder(monoid).Build();

  auto right_ideals = IdealsBuilder(monoid, condensation_graph).Build();

  ASSERT_THAT(IndicesToWords(monoid, right_ideals),
              Eq(std::vector<std::vector<std::string>>{
                  {"a", "b", "c", "ab", "bb", "ca"}, {"a", "c", "ab", "ca"}, {"c", "ca"}}));
}
  \end{minted}
\end{listing}

\newpage
\anonsection{ЗАКЛЮЧЕНИЕ}

В результате работы получена реализация алгоритмов на стыке алгебры и дискретной
математики. Цели практики достигнуты: программа протестирована модульными
тестами и соответствует заданию.

Выполняя данную работу, я подтянул знания алгебры и узнал о ее связи с
дискретной математикой, используя книгу <<Mathematical foundations of Automata
Theory>>~\cite{mathfoundations2025}. Также мне выпала приятная возможность
применить на практике фундаментальные алгоритмы из области генерации
комбинаторных объектов и теории графов.

\renewcommand\refname{СПИСОК ИСПОЛЬЗОВАННЫХ ИСТОЧНИКОВ}
% Список литературы
\clearpage

%\bibliographystyle{ugost2008s}  %utf8gost71u.bst} %utf8gost705u} %gost2008s}
{\catcode`"\active\def"{\relax}
\addcontentsline{toc}{section}{\protect\numberline{}\refname}%
%\bibliography{biblio} %здесь ничего не меняем, кроме, возможно, имени bib-файла
\printbibliography
}

\newpage
\settocdepth{section}
\anonsection{ПРИЛОЖЕНИЕ А}

\vspace{-30pt}

\begin{listing}[H]
\caption{Построение графа конденсации.}
\label{lst:condensation}
\begin{algorithmic}[1]
\Procedure{Condensation}{$G(V, E)$}
    \State $\text{used} \gets \text{false}\quad{} \forall v \in V$
    \State $\text{order} \gets \{\}$

    \For{$v \in V$}
        \If{$\text{used}[v] = \text{false}$}
            \State $\text{DFS1}(v)$
        \EndIf
    \EndFor

    \State $G^T \gets \text{Transposed(G)}$
    \State $\text{used} \gets \text{false}\quad{} \forall v \in V$
    \State $\text{components} \gets \{\}$

    \For{$v \in \text{Reversed(order)}$}
        \If{$\text{used}[v] = \text{false}$}
            \State $\text{comp} \gets \{\}$
            \State $\text{DFS2}(v, \text{comp})$
            \State $\text{components} \gets \text{comp}$
        \EndIf
    \EndFor

    \State $C \gets (\{\}, \{\})$
    \For{$c_i \in \text{components}$}
        \State $C.V \gets c_{i}$
    \EndFor
    \For{$(u, v) \in E$}
        \State $c_u \gets$ Component with $u$
        \State $c_v \gets$ Component with $v$
        \If{$c_u \neq c_v$}
            \State $C.E \gets (c_u, c_v)$
        \EndIf
    \EndFor
    \State \Return $C$
\EndProcedure

\Function{DFS1}{$v$}
    \State $\text{used}[v] \gets \text{true}$
    \For{$u \in G.adj[v]$}
        \If{$\text{used}[u] = \text{false}$}
            \State $\text{DFS1}(u)$
        \EndIf
    \EndFor
    \State $\text{order} \gets v$
\EndFunction

\Function{DFS2}{$v$, $\text{comp}$}
    \State $\text{used}[v] \gets \text{true}$
    \State $\text{comp} \gets v$
    \For{$u \in G^T.adj[v]$}
        \If{$\text{used}[u] = \text{false}$}
            \State $\text{DFS2}(u, \text{comp})$
        \EndIf
    \EndFor
\EndFunction
\end{algorithmic}
\end{listing}

\end{document}
